

Vektorin pituus saadaan laskettua Pythagoraan lauseen avulla.  (KUVA) Esimerkiksi vektorin $\va = -2\vi + 3\vj$ pituus saadaan yht�l�st�
\[
|\va|^2 = 2^2 + 3^2.
\]
Vektorin $\va$ pituudeksi saadaan $|\va| = \sqrt{2^2 + 3^2}$.

\begin{teht}\label{tehtava:yksikkovektori1}
Tutkitaan vektoria $\vb = 3\vi - 4\vj$. 
\begin{aakkosta*}
\item Piirr� vektori $\vb$ koordinaatistoon.
\item Laske vektorin $\vb$ pituus $|\vb|$ Pythagoraan lauseen avulla.
\item Kuinka moneen osaan vektori $\vb$ pit�isi jakaa, jotta yhden osan pituus olisi $1$?
\end{aakkosta*}
\end{teht}


\maaritelma[Yksikk�vektori]{Vektoria, jonka pituus on $1$, sanotaan yksikk�vektoriksi.} 

Esimerkiksi vektorin $\vv = 8\vi + 6\vj$ pituudeksi saadaan $|\vv| = \sqrt{8^2+6^2} = \sqrt{100} = 10$. Sen kanssa samansuuntainen yksikk�vektori saadaan ottamalla vektorista $\vv$ kymmenesosa eli 
\[\frac{1}{10}\vv =  \frac{1}{10}(8\vi + 6\vj) = 0{,}8\vi + 0{,}6\vj\]
(KUVA)

\begin{teht}
Jatkoa teht�v��n \ref{tehtava:yksikkovektori1}. Tutkitaan edelleen vektoria $\vb = 3\vi - 4\vj$. 
\begin{aakkosta*}
\item M��rit� vektorin $\vb$ kanssa samansuuntainen yksikk�vektori eli vektori, joka pituus on $1$. Piirr� se koordinaatistoon.
\item M��rit� vektorin $\vb$ kanssa samansuuntainen vektori, jonka pituus on $10$. Piirr� se koordinaatistoon.
\item M��rit� vektorin $\vb$ kanssa vastakkaissuuntainen yksikk�vektori. Piirr� se koordinaatistoon.
\end{aakkosta*}
\end{teht}







