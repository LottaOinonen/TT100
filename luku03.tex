\section{Vektorien geometrisia sovelluksia}



\subsection{Yleinen vektori}

\textsc{t�ss� kerrataan lukujen 1 ja 2 tietoja ilman koordinaatistoa.}

Tutkitaan alla olevaa kuvaa \ref{kuva:yleinen vektori intro}. Huomataan, ett� vektori $\vv$ voidaan ilmaista vektoreiden $\va$ ja $\vb$ avulla muodossa $\vv=\va+2\vb$. 

	\begin{center}
	\begin{figurehere}
	\begin{tikzpicture}[line cap=round,line join=round,>=triangle 45,x=0.7cm,y=0.7cm]

		% Ruudukko
		\foreach \x in {-5,-4,-3,-2,0,-1,1,2,3,4,5}
		{
		\draw[dashed, color= lightgray] (\x,-3.5)--(\x,3.5);
		}
		
		\foreach \y in {-3,-2,-1,0,1,2,3}
		{
		\draw[dashed, color=lightgray] (-5.5,\y)--(5.5,\y);
		}

		% Varsinainen kuva
		
		% Vektorit
		
		\draw[->] (-5,0)--(-1,1);
		\node[above] at (-3,1) {$\va$};
		
		\draw[->] (-5,1)--(-4,3);
		\node[above left] at (-5,2) {$\vb$};
		
		\draw[->] (-2,-2)--(4,3);
		\node[above] at (1,1) {$\vv$};
		
		\draw[->, color=gray] (-2,-2)--(2,-1);
		\node[below] at (0,-1.5) {\textcolor{darkgray}{$\va$}};
		
		\draw[->, color=gray] (2,-1)--(3,1);
		\node[below right] at (3,1) {\textcolor{darkgray}{$\vb$}};
		
		\draw[->, color=gray] (3,1)--(4,3);
		\node[below right] at (3.5,2.5) {\textcolor{darkgray}{$\vb$}};	

	\end{tikzpicture}
	\caption{Vektori $\vv = \va + 2\vb$.}
	\label{kuva:yleinen vektori intro}
	\end{figurehere}
	\end{center}


\begin{teht} Tutkitaan seuraavaa kuvaa \ref{kuva:yleisten vektorien laskutoimituksia}.

	\begin{center}
	\begin{figurehere}
	\begin{tikzpicture}[line cap=round,line join=round,>=triangle 45,x=0.7cm,y=0.7cm]

		% Ruudukko
		\foreach \x in {-5,-4,-3,-2,0,-1,1,2,3,4,5}
		{
		\draw[dashed, color= lightgray] (\x,-3.5)--(\x,3.5);
		}
		
		\foreach \y in {-3,-2,-1,0,1,2,3}
		{
		\draw[dashed, color=lightgray] (-5.5,\y)--(5.5,\y);
		}

		% Varsinainen kuva
		
		% Vektorit
		
		\draw[->] (-5,0)--(-1,1);
		\node[above] at (-3,1) {$\va$};
		
		\draw[->] (-5,1)--(-4,3);
		\node[above left] at (-5,2) {$\vb$};
		
		\draw[->] (-2,-2)--(4,3);
		\node[above] at (1,1) {$\vv$};
		
		\draw[->, color=gray] (-2,-2)--(2,-1);
		\node[below] at (0,-1.5) {\textcolor{gray}{$\va$}};
		
		\draw[->, color=gray] (2,-1)--(3,1);
		\node[below right] at (3,1) {\textcolor{gray}{$\vb$}};
		
		\draw[->, color=gray] (3,1)--(4,3);
		\node[below right] at (3.5,2.5) {\textcolor{gray}{$\vb$}};	

	\end{tikzpicture}
	\caption{Vektorit $\vv$, $\vw$ ja $\vu$.}
	\label{kuva:yleisten vektorien laskutoimituksia}
	\end{figurehere}
	\end{center}

\begin{aakkosta*}
\item Piirr� vektori $\vv+\vw$.
\item Piirr� vektori $\vv-\vu$.
\item Piirr� vektori $-\vv+\vw+\vu$.
\end{aakkosta*}
\end{teht}

\begin{teht} Tarkastellaan alla olevaa kuvaa \ref{}. Ilmaise vektorien $\va$ ja $\vb$ avulla vektori 
\begin{aakkosta*}
\item $\vv$  
\item $\vw$
\item $\vu$.
\end{aakkosta*}
\textsc{Harjoitellaan sit�, ett� muiden vektoreiden kuin i ja j avulla. vektorit ovat kaikki vektorien $\va$ ja $\vb$ lin. komb.}
\end{teht}

\begin{teht} Kuvassa \ref{} on vektorit $\va$ ja $\vb$. Piirr� vektori $-1/2\vb+2/3\va$.
\textsc{$\va=(3,3)$ ja $\vb=3,6$}
\end{teht}

\begin{teht} Vektorin $\va$ pituus on $8$ ja vektorin $\vb$ 6. Ilmaise vektori $\vb$ vektorin $\va$ avulla, jos vektorit $\va$ ja $\vb$ ovat
\begin{aakkosta*}
\item samansuuntaiset
\item vastakkaissuuntaiset.
\end{aakkosta*}
\end{teht}

\begin{teht} Oletetaan, ett� $\va, \vb \neq \bar{0}$. Vertaa vektoreiden $va$ ja $vb$ pituuksia ja suuntia kesken��n, kun
\begin{aakkosta*}
\item $\va-2\vb=2\va+\vb$
\item ...
\item ...
\end{aakkosta*}
\end{teht}





\subsection{Reittej�}

\textsc{t�ss� opitaan muodostamaan reittej� vektorien avulla.}

Vektorien avulla voidaan muodostaa kuvioita. Tarkastellaan vektoreita $\va=$, $\vb=$ ja $\vc=$. Kun kuljetaan ensin $2\va+\vb-3\vc$ muodostuu kolmio, joka on esitetty kuvassa \ref{}.

KUVA 

\begin{teht} Kuvassa \ref{} on vektorit $\va, \vb, \vc, \vv, \vw$ ja $\vu$. Mik� kuvio muodostuu, kun kuljetaan reitti
\begin{aakkosta*}
\item joku reitti
\item joku toinen reitti
\end{aakkosta*}
\end{teht}


\begin{teht}
\begin{aakkosta*}
\item
\item
\end{aakkosta*}
\end{teht}

\begin{teht}
\begin{aakkosta*}
\item
\item
\end{aakkosta*}
\end{teht}


\subsection{Geometriaa vektoreilla}

\textsc{t�ss� opitaan geometriaan liittyvi� asioita: ''piste jakaa janan suhteessa...'', keskijanavektori jne.}

Esimerkki. Olkoon AP=1/3AB. Piste P jakaa janan AB suhteessa? 

\begin{teht} Miss� suhteessa piste $P$ jakaa janan $AB$, kun 
\begin{aakkosta*}
\item $\pv{AP}=3/5\pv{AB}$
\item $\pv{AB}=7/4\pv{AP}$.
\end{aakkosta*}
\end{teht}


\subsection{Vektorit k�yt�nn�ss�}

\textsc{K�yt�nn�n el�m�n tilanteita vektoreilla.}